%---------- Inleiding ---------------------------------------------------------

% TODO: Is dit voorstel gebaseerd op een paper van Research Methods die je
% vorig jaar hebt ingediend? Heb je daarbij eventueel samengewerkt met een
% andere student?
% Zo ja, haal dan de tekst hieronder uit commentaar en pas aan.

%\paragraph{Opmerking}

% Dit voorstel is gebaseerd op het onderzoeksvoorstel dat werd geschreven in het
% kader van het vak Research Methods dat ik (vorig/dit) academiejaar heb
% uitgewerkt (met medesturent VOORNAAM NAAM als mede-auteur).
% 

\section{Inleiding}%
\label{sec:inleiding}

De bachelorproef richt zich op het gebruik van Large Language Models (LLM's) om mensen met alexithymie te ondersteunen bij het identificeren en verwoorden van hun emoties. Alexithymie is een aandoening waarbij individuen moeite hebben met het herkennen en beschrijven van hun eigen emoties, wat kan leiden tot problemen in persoonlijke relaties en therapieën. Het doel van dit onderzoek is om te onderzoeken hoe LLM's, zoals GPT-4, kunnen worden ingezet om deze doelgroep te helpen door middel van gepersonaliseerde vragenlijsten en oefeningen die zijn gebaseerd op bestaande therapeutische methoden zoals cognitieve gedragstherapie (CBT).

De centrale probleemstelling van dit onderzoek is: *Hoe kunnen Large Language Models worden gebruikt om mensen met alexithymie te ondersteunen bij het identificeren en verwoorden van hun emoties?* De onderzoeksdoelstelling is om een platform te ontwikkelen dat gebruik maakt van een LLM om mensen met alexithymie te begeleiden in het herkennen van hun emoties, met als eindresultaat een proof-of-concept dat de effectiviteit van deze aanpak aantoont. Dit platform zal worden getest bij een groep deelnemers die positief scoren op de Toronto Alexithymia Scale (TAS-20), waarbij we zowel objectieve verbeteringen in emotieherkenning als subjectieve gebruikerservaringen zullen evalueren.

%---------- Stand van zaken ---------------------------------------------------

\section{Literatuurstudie}%
\label{sec:literatuurstudie}

\subsection{Mapping van Emoties op het lichaam}
\label{subsec:mapping-van-emoties-op-het-lichaam}%

Onderzoek toont aan dat emoties onder andere worden ervaren als lichamelijke sensaties in specifieke delen van het lichaam. \textcite{Nummenmaa2013} hebben met behulp van topografische zelfrapportage methoden aangetoond dat verschillende emoties consistent geassocieerd zijn met unieke sensaties in het lichaam. Zo wordt woede bijvoorbeeld vaak gevoeld in het bovenlijf, terwijl verdriet gepaard gaat met verminderde activiteit over heel het lichaam. Deze bevindingen suggereren dat emoties niet alleen mentale toestanden zijn, maar ook lichamelijke ervaringen die op het lichaam kunnen worden gemapt.

Deze inzichten kunnen nuttig zijn voor Large Language Models (LLM's) om mensen met alexithymie te ondersteunen bij het identificeren van hun emoties. Door de mapping van emoties op het lichaam, zoals die door \textcite{Nummenmaa2013} zijn vastgesteld, te converteren naar een formaat dat begrijpelijk is voor een LLM, kan het model vragen stellen over waar de gebruiker sensaties ervaart in het lichaam en op basis van deze informatie kan de LLM vervolgens suggesties doen over welke emoties mogelijk worden ervaren, hierdoor kunnen gebruikers geholpen worden bij het herkennen en verwoorden van hun gevoelens, ondanks hun moeilijkheid om deze zelf te identificeren \autocite{Nummenmaa2013}.

\subsection{Verhalen en afbeeldingen voor het trainen van Emotieherkenning}
\label{subsec:verhalen-en-afbeeldingen-voor-het-trainen-van-emotieherkenning}%

In de studie van \textcite{Lukas2019} werd een smartphone-gebaseerde interventie ontwikkeld om emotieherkenningsvaardigheden te trainen bij mensen met alexithymie. De interventie combineerde psycho-educatie met een 14-daagse training via de app MT-ALEX, waarbij gebruikers verhalen lazen en deze koppelden aan tekstuele beschrijvingen en afbeeldingen om emoties te herkennen. De resultaten toonden aan dat deze aanpak leidde tot een significante verbetering in emotieherkenning \autocite{Lukas2019}.

Het gebruik van verhalen en afbeeldingen kan dus nuttig zijn om mensen met alexithymie te helpen bij het herkennen van emoties. Large Language Models (LLM's) kunnen op een vergelijkbare manier worden ingezet, door zowel AI-gegenereerde verhalen als persoonlijke ervaringen van de gebruiker te gebruiken. Bovendien kunnen LLM's worden gekoppeld aan technologieën die afbeeldingen genereren of ophalen, wat een geïntegreerde aanpak mogelijk maakt om emotieherkenning te bevorderen \autocite{Lukas2019}.

\subsection{Gedragstherapie voor alexithymie}
\label{subsec:gedragstherapie-voor-alexithymie}%

In de studie van \textcite{Cameron2014} wordt besproken hoe verschillende psychologische interventies, zoals cognitieve gedragstherapie (CBT) en psychodynamische therapie, effectief kunnen zijn in het verminderen van alexithymie. Deze therapieën richten zich vaak op het verbeteren van emotionele herkenning en expressie, wat cruciaal is voor mensen met alexithymie. De studie concludeert dat alexithymie gedeeltelijk kan worden verminderd door therapeutische interventies, vooral wanneer deze interventies specifiek gericht zijn op de symptomen van alexithymie \autocite{Cameron2014}.

Op basis van bestaande therapiekaders zoals CBT kunnen Large Language Models (LLM's) worden ingezet om gepersonaliseerde vragenlijsten en oefeningen te ontwikkelen die gebruikers helpen hun emoties beter te identificeren en te verwoorden. Door deze therapieën te vertalen naar een digitaal format, kan een LLM niet alleen vragen stellen, maar ook feedback geven op de antwoorden van de gebruiker, wat een waardevolle aanvulling kan zijn op traditionele therapieën \autocite{Cameron2014}.

\subsection{Emotieherkenning door Large Language Models}
\label{subsec:emotieherkenning-door-large-language-models}%

In een recente studie over emotieherkenning door Large Language Models (LLM's) werd onderzocht hoe goed deze modellen emoties kunnen identificeren en beschrijven. De studie vergeleek de prestaties van verschillende modellen, waaronder GPT-3.5, GPT-4 en Bard, op basis van de Toronto Alexithymia Scale (TAS-20) en de Empathy Quotient (EQ-60). Uit de resultaten bleek dat GPT-4 het dichtst bij het menselijke niveau van emotieherkenning kwam, terwijl andere modellen zoals GPT-3.5 en Bard slechter presteerden, met name op het gebied van alexithymie \autocite{Patel2023}.

Deze bevindingen hebben ons doen besluiten om GPT-4 te gebruiken in onze studie, gezien zijn superieure vermogen om emoties te herkennen binnen geschreven taal. Dit model biedt echter uitdagingen op het gebied van reproduceerbaarheid en privacy. Aangezien GPT-4 een commercieel model is dat niet publiekelijk beschikbaar is, kan dit problemen opleveren voor wetenschappelijke herhaalbaarheid. Daarnaast is er een privacyrisico, omdat OpenAI toegang heeft tot de gesprekken die via hun model worden gevoerd. Gebruikers moeten daarom expliciet geïnformeerd worden over deze risico's \autocite{Patel2023}.



% Voor literatuurverwijzingen zijn er twee belangrijke commando's:
% \autocite{KEY} => (Auteur, jaartal) Gebruik dit als de naam van de auteur
%   geen onderdeel is van de zin.
% \textcite{KEY} => Auteur (jaartal)  Gebruik dit als de auteursnaam wel een
%   functie heeft in de zin (bv. ``Uit onderzoek door Doll & Hill (1954) bleek
%   ...'')

%---------- Methodologie ------------------------------------------------------
\section{Methodologie}%
\label{sec:methodologie}

\subsection{1. Voorbereiding}
Alle relevante informatie uit de literatuur wordt verzameld en omgezet naar functionele vereisten voor het systeem. Dit omvat de capaciteiten die het Large Language Model (LLM) moet hebben om mensen met alexithymie te ondersteunen bij het identificeren en verwoorden van emoties.

\subsection{2. Dataverzameling en verwerking}
Er wordt specifieke data verzameld over:
\begin{itemize}
  \item Mapping van emoties op het lichaam.
  \item Definities en beschrijvingen van emoties.
  \item Informatie en testen gerelateerd aan therapeutische methoden zoals cognitieve gedragstherapie (CBT) en psychodynamische therapie.
\end{itemize}
Deze data wordt omgezet naar een formaat dat door het LLM kan worden verwerkt. Prompts worden ontwikkeld en getest voor gebruik binnen het platform.

\subsection{3. Platform bouwen}
Het platform wordt gebouwd met behulp van de OpenAI API, waarbij een back-end in Python wordt geschreven. Python is gekozen vanwege zijn toegankelijkheid en brede gebruik in toepassingen met LLM's. Het platform zal dienen als een interface voor gebruikers.

\subsection{4. Deelnemers werven}
Deelnemers worden gerekruteerd en leggen de Toronto Alexithymia Scale (TAS-20) af, een standaardtest voor alexithymie. Deelnemers die boven een bepaalde drempel scoren, worden geselecteerd voor deelname aan het experiment.

\subsection{5. Experiment uitvoeren}
Gedurende 14 dagen spenderen deelnemers dagelijks minstens 15 minuten op het platform. Geanonimiseerde data over hun interacties wordt verzameld, en na afloop leggen ze opnieuw de TAS-20 test af om verbeteringen te meten. Daarnaast worden hun subjectieve ervaringen geëvalueerd.

\subsection{6. Data-analyse}
De verzamelde data wordt geanalyseerd om te bepalen of het gebruik van het platform heeft geleid tot verbeteringen in emotieherkenning bij gebruikers met alexithymie. Objectieve resultaten worden vergeleken met hun subjectieve ervaringen om correlaties te identificeren.

\subsection{7. Rapportage}
De bevindingen van dit onderzoek worden gerapporteerd om te evalueren of het platform effectief is geweest in het ondersteunen van gebruikers bij het identificeren en verwoorden van hun emoties.

%---------- Verwachte resultaten ----------------------------------------------
\section{Verwacht resultaat, conclusie}%
\label{sec:verwachte_resultaten}

We verwachten dat het platform een significante verbetering zal laten zien in het vermogen van deelnemers met alexithymie om emoties te herkennen en te verwoorden. Deelnemers die het platform gebruiken, zullen naar verwachting lagere scores behalen op de Toronto Alexithymia Scale (TAS-20) na de interventie, wat wijst op een vermindering van alexithymische kenmerken. Daarnaast verwachten we dat gebruikers positieve feedback zullen geven over de bruikbaarheid van het platform en hoe het hen heeft geholpen bij het begrijpen van hun emoties.

